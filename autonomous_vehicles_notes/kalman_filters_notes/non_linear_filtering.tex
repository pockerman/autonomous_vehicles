\section{Non-linear Filtering}
\label{non_linear_filtering}

We developed the Kalman filter algorithm using the following linear state space model

\begin{eqnarray}
\mathbf{x}_k = \mathbf{F}_{k-1}\mathbf{x}_{k-1} + \mathbf{G}_{k-1}\mathbf{u}_{k-1} + \mathbf{w}_{k-1} \\
\mathbf{y}_k = \mathbf{H}_k \mathbf{x}_k + \mathbf{v}_k
\end{eqnarray}

Using this model, we then soleved the filtering problem iteratively using a predictor-corrector type of scheme

\begin{eqnarray}
\text{Predict:}~~ p(\mathbf{x}_k|\mathbf{y}_{1:k-1}) = \int p(\mathbf{x}_k|\mathbf{x}_{k-1})p(\mathbf{x}_{k-1}|\mathbf{y}_{1:k-1}) d \mathbf{x}_{k-1} \\
\text{Update:}~~ p(\mathbf{x}_k|\mathbf{y}_{1:k}) = \frac{p(\mathbf{y}_k|\mathbf{x}_{k})p(\mathbf{x}_{k}) | \mathbf{y}_{1:k-1})}{\int p(\mathbf{y}_k|\mathbf{x}_{k})p(\mathbf{x}_{k}|\mathbf{y}_{1:k-1}) d \mathbf{x}_{k}}
\end{eqnarray}

For linear Gaussian models the Kalman filter provides and analytical solution.


\subsection{Nonlinear models}

However, many important motion and measurement models are nonlinear. For example the coordinated turn motion model is given by

\begin{eqnarray}
\begin{bmatrix}
x_k \\
y_k \\
v_k \\
\phi_k \\
\omega_k
\end{bmatrix} = 
\begin{bmatrix}
x_{k-1} + \frac{2v_{k-1}}{\omega_{k-1}}\sin(\frac{\omega_{k-1}T)}{2}\cos(\phi_{k-1} + \frac{\omega_{k-1}T}{2}) \\
y_{k-1} + \frac{2v_{k-1}}{\omega_{k-1}}\sin(\frac{\omega_{k-1}T)}{2}\sin(\phi_{k-1} + \frac{\omega_{k-1}T}{2}) \\
v_{k-1}
\phi_{k-1} + T \omega_{k-1} \\
\omega_{k-1}
\end{bmatrix} + \mathbf{q}_{k-1} \\
y_k = \sqrt{x_{k}^2 + y_{k}^2} + r_k
\end{eqnarray}

for such models we, generally, cannot obtain an anlytical solution to the filtering equations. A common approach to solve this, is to find tractable approximate solutions.


\begin{framed}
\theoremstyle{remark}
\begin{remark}{\textbf{Why is nonlinear filtering hard?}}

This is because of the following two reasons

\begin{itemize}
\item We cannot in general find an analytical solution to our filtering equations.
\item For linear systems, it is enough to describe the posterior using the first two moments, the mean and the covariance. For nonlinear systems, 
we in general need infinitely many moments to describe the true posterior.
\end{itemize}

The optimal filtering equations hold also in the non-linear setting. It is just harder to solve the equations, because we cannot find 
analytical solutions to them (in general). When the models are linear and Gaussian, the posterior density 
is also Gaussian, which means it could be described completely by the first two moments only. 
In the non-linear setting, we need infinitely many moments to fully describe the posterior, so approximations are necessary.

Weather the Markov property holds or not decides if you can at all use recursive filters to solve your estimation problem. 
In this chapter we still assume that the Markov property holds.
\end{remark}
\end{framed} 

Nonlinear filtering can be a difficult and complex subject.  It is certainly not as 
mature, cohesive, or well understood  as linear filtering. There is still a lot of  room 
for advances and improvement in nonlinear estimation techniques.  However, some 
nonlinear  estimation methods have become  (or are becoming) widespread.  These 
techniques  include nonlinear  extensions of  the Kalman filter,  unscented  filtering, 
and particle filtering. 


\section{Questions}

\begin{enumerate}
\item Which of the following statements are true?

\begin{enumerate}
\item Our choice of density parametrization will both effect the performance of our filter and the computational complexity. 
\item In real-time applications, having a recursive filter means that our compute time for producing a new estimate is more or less constant with each new measurement. 
\item In nonlinear filtering, a recursive algorithm will always find a better or equally good approximation of the true posterior as a non-recursive filter. 
\item A prerequisite for a recursive filter is that the prior and posterior density in each recursion can be described using the same density parametrization. 
\end{enumerate}
\end{enumerate}


