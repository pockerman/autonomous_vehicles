\section{Extended Kalman Filter}
\label{extended_kalman_filter}

All of  our discussion to this point has considered linear filters for linear systems.
Unfortunately, linear systems do not  exist.  All systems are ultimately nonlinear.

However, many systems are  close enough to linear that linear estimation approaches give satisfactory results. 
But  close enough can only be carried so far.  Eventually, we run across a system 
that does not behave linearly even over a small range of  operation, and our linear 
approaches for  estimation  no longer give good results.  In this case, we  need  to  explore nonlinear estimators. 

Nonlinear filtering can be a difficult and complex subject.  It is certainly not as 
mature, cohesive, or well understood  as linear filtering. There is still a lot of  room 
for advances and improvement in nonlinear estimation techniques.  However, some 
nonlinear  estimation methods have become  (or are becoming) widespread.  These 
techniques  include nonlinear  extensions of  the Kalman filter,  unscented  filtering, 
and particle filtering.

In this section, we will discuss some nonlinear extensions of  the Kalman filter. 
The Kalman filter  that we  discussed earlier  in this book  directly  applies only to 
linear  systems.  However,  a  nonlinear  system  can  be  linearized  as discussed  in 
Section  1.3, and then  linear  estimation  techniques  (such  as the  Kalman  or  $H_{\infty}$, 
filter) can be applied.  This chapter discusses those types of approaches to nonlinear 
Kalman filtering.



\subsection{The Linearized Kalaman Filter }
In  this section, we  will show how  to linearize a nonlinear system, and  then  use 
Kalman  filtering theory to estimate the  deviations  of  the state  from a  nominal 
state value. This will then give us an estimate of the state of the nonlinear system.

We will derive the linearized Kalman filter from the continuowtime viewpoint, but 
the analogous derivation for discretetime or hybrid systems are straightforward.

\subsection{The Extended Kalaman Filter }

The previous section obtained a linearized Kalman filter for estimating the states of 
a nonlinear system.  The derivation was based on linearizing the nonlinear system 
around a nominal state trajectory.  The question that arises is, How do we know 
the nominal state trajectory? 

In some cases it may not be straightforward to find 
the nominal trajectory.  However, since the Kalman  filter estimates the state of 
the system, we can use the Kalman filter estimate as the nominal state trajectory.

This is sort of a bootstrap method.  We linearize the nonlinear system around the 
Kalman filter estimate, and the Kalman filter estimate is based on the linearized 
system. This is the idea of the extended Kalman filter (EKF), which was originally 
proposed by Stanley Schmidt so that the Kalman filter could be applied to nonlinear 
spacecraft navigation problems.
