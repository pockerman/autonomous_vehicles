\section{Vehicle Modeling with Chronos}
\label{vehicle_modeling_chronos}

In this section we will develop and simulate vehicle model using the open source physics engine \lstinline{chronos} 


\subsection{The \lstinline{Chrono::Vehicle} library}

The \lstinline{Chrono::Vehicle} is a C++ middleware library for the modeling, simulation, and visualization of wheeled and tracked ground vehicles.
It consists of two core modules:

\begin{itemize}
\item The \lstinline{ChronoEngine_vehicle}

	\begin{itemize}
		\item Defines the system and subsystem base classes
		\item Provides concrete, derived classes for instantiating templates from JSON specification files
		\item Provides miscellaneous utility classes and free functions for file I/O, Irrlicht vehicle visualization, steering and speed controllers, vehicle and subsystem test rigs, etc.
	\end{itemize}

\item The \lstinline{ChronoModels_vehicle}
	\begin{itemize}
		\item Provides concrete classes for instantiating templates to model specific vehicle models
	\end{itemize}
\end{itemize}

The following dependencies should be satisfied in order to use the library.

\begin{itemize}
\item The \lstinline{Chrono::Engine } required
\item The \lstinline{Chrono::Irrlicht} and the \lstinline{Irrlicht} library,  \lstinline{Chrono::OpenGL} and its dependencies. Both are optional
\item The \lstinline{Chrono::FEA} and \lstinline{Chrono::MKL} (optional)
\end{itemize}

The \lstinline{Chrono::Engine } supports the notion of a system. In our case, the following components are considered a system

\begin{itemize}
\item Powertrain
\end{itemize}

\lstinline{Chrono::Vehicle} encapsulates templates for systemsand subsystems in polymorphic C++ classes:

\begin{itemize}
\item A base abstract class for the system/subsystem type (e.g. \lstinline{Chrono::ChSuspension})
\item A derived, still abstract class for the system/subsystem template (e.g.  \lstinline{Chrono::ChDoubleWishbone})
\item Concrete class that particularize a given system/subsystem template (e.g. \lstinline{Chrono::HMMWV_DoubleWishboneFront})
\end{itemize}

\subsection{Setup simulation}

\subsubsection{Setup the vehicel \lstinline{chrono::vehicle::sedan::Sedan}}

Now that we went over the basics of the \lstinline{Chrono::Vehicle} library let's try to set up a basic simulation; namely a vehicle that move in straight line.
Concretely, we will use an instance of the \lstinline{chrono::vehicle::sedan::Sedan} class. The following code initializes the vehicle instance for the simulation

\begin{lstlisting}
// Create the vehicle, set parameters, and initialize
    Sedan vehicle;
    vehicle.SetContactMethod(contact_method);
    vehicle.SetChassisFixed(false);
    vehicle.SetInitPosition(ChCoordsys<>(initLoc, initRot));
    
    vehicle.SetTireType(tire_model);
    vehicle.SetTireStepSize(tire_step_size);
    vehicle.SetVehicleStepSize(step_size);
    vehicle.Initialize();

    vehicle.SetChassisVisualizationType(chassis_vis_type);
    vehicle.SetSuspensionVisualizationType(suspension_vis_type);
    vehicle.SetSteeringVisualizationType(steering_vis_type);
    vehicle.SetWheelVisualizationType(wheel_vis_type);
    vehicle.SetTireVisualizationType(tire_vis_type);
\end{lstlisting}


\subsubsection{Create the application}

\begin{lstlisting}
// Create the vehicle Irrlicht application
ChVehicleIrrApp app(&vehicle.GetVehicle(), &vehicle.GetPowertrain(), 
		    L"Steering XT Controller Demo", 
        irr::core::dimension2d<irr::u32>(800, 640));

app.SetHUDLocation(500, 20);
app.SetSkyBox();
app.AddTypicalLogo();

irr::core::vector3df v1(-150.f, -150.f, 200.f);
irr::core::vector3df v2(-150.f, 150.f, 200.f);
irr::core::vector3df v3(150.f, -150.f, 200.f);
irr::core::vector3df v4(150.0f, 150.f, 200.f); 
app.AddTypicalLights(v1, v2, 100, 100);
app.AddTypicalLights(v3, v4, 100, 100);
app.EnableGrid(false);
app.SetChaseCamera(trackPoint, 6.0, 0.5);
app.SetTimestep(step_size);
\end{lstlisting}

%So, our robot can roll forward and turn while rolling, but cannot move sideways directly. We'll use this constraint to define a kinematic model for our robot. The velocity of the robot is defined by the tangent vector to its path see figure \ref{vehicle_path}. 

%\begin{figure}[!htb]
%\begin{center}
%\includegraphics[scale=0.290]{img/kinematics/vehicle_path.jpeg}
%\end{center}
%\caption{Vehicle velocity.}
%\label{vehicle_path}
%\end{figure}

The following link can be used to consult for further information \lstinline{http://api.projectchrono.org/tutorial_install_project.html}












